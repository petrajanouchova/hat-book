\environment env_dis

\starttext

\startchapter[title={About \index{the} the project},reference={about}]

As my digital project I will set up \index{and} and test typesetting production environment for my dissertation. I will use \ConTeXt, running on 
\index{Linux} Linux Ubuntu 14.04. I will also set up GitHub project to keep my work somewhere transparent, but safe.

\section[section1]{Step 1: planning, 29 April}

Here is \index{the} the timeframe of my work, as discussed during FOAR705 on 29 April 2016. I \index{have} have allocated 40-60 hours of my time: ideally 40 hours on set up \index{and} and testing, \index{and} and additional 20 hrs for potential problems that may arise.

I \index{have} have set up my internal deadline as 31 May, but ideally \index{the} the actual work should be done around 15 May \index{and} and \index{the} the extra two weeks give me some space for unexpected changes (illness, software or hardware issues, o\index{the} ther project deadlines, personal life issues).

I \index{have} have done my planning very granular, \index{because} because that is how I work. I like small steps, careful planning \index{and} and a lot of time (back up) for any unexpected events, such as me getting scarlet fever in \index{the} the middle of \index{the} the thing or breaking my vertebra when getting up in \index{the} the morning (been \index{the} there, done that).

Moreover, I \index{have} have thought \index{the} the steps ahead, so I had no problems articulating \index{the} them quite clearly even with \index{the} the timeframe allocated to individual elements. We will see how it goes. In general, everything takes me longer than I planned it, but I tried to be generous this time.

\startitemize[2]
\item Allocated/Spent time: 3/3 h, during \index{the} the FOAR705
\stopitemize

\subsection[technical]{Technical planning}
As part of planning, I \index{have} have talked to Brian about technical part of my project. We \index{have} have discussed \index{the} the technical options: \ConTeXt is best running on 
\index{Linux} Linux, Brian suggested 14.04. As I \index{have} have very old computer, I can’t do Virtual Machine or Disk Partition. I am in final stages of my PhD \index{and} and I don’t want to lose programs such as  Adobe Acrobat, Photoshop, Bridge, Illustrator etc., so I cannot entirely kill my Windows OS. I \index{have} have \index{the} the following options:
\startitemize[packed,1]
\item Run \ConTeXt on Windows, which will likely be problematic (risk: spending too much time on it with no result, \index{and} and growing frustration; my decision: tried to run it, but even \index{the} the installing gave me too many errors so I decided to ab\index{and} andon this scenario)
\item Buy a new computer ASAP \index{and} and run Virtual Machine \index{the} there, or switch completely to 
\index{Linux} Linux (risk: money money money. For \index{the} the type of laptop I want, \index{the} the price start around 1600 AUD; my decision: will try to find ano\index{the} ther option first, but in \index{the} the worst case I will \index{have} have to do this)
\item Borrow my friend’s old computer, try to run it Ubuntu \index{the} there \index{and} and see what happens (risk: spending time on it, with no or minimal result, \index{and} and having to buy a new laptop after all; my decision: worth trying first)
\stopitemize

\startitemize[2]
\item Allocated/Spent time: 2/2 h (1 h talking to Brian, 1h my own research)
\stopitemize

\subsection[instructions]{Instructions for \index{the} the final product}

I \index{have} have spent some time looking for instruction how \index{the} the final product (dissertation) should look like \index{and} and as a result I \index{have} have found out that I am allowed to do almost anything I want. \index{the} there are some  basic rules issued by my faculty how \index{the} the title page should look like, which font size to use etc., but that’s it.

\startitemize[2]
\item Allocated/Spent time: 0.5/1h (took me longer to find out, \index{the} there are no definite rules! I had to manually download dissertations from our department from last 10 years to discover \index{the} there are no universal rules).
\stopitemize

\section[section2]{Software preparation, 30 April}

I \index{have} have decided for scenario C: borrow my friend’s old computer \index{and} and install Ubuntu 14.04 on it, document to whole process in detail \index{and} and \index{the} then when I acquire my own new computer, recreate \index{the} the environment \index{the} there.

\subsection{Ubuntu 14.04 install}
\index{the} the laptop I am using: Lenovo X201, Intel Core I7 processor I7, currently disk partition of Windows 7 \index{and} and Ubuntu 14.04. My task is to get rid of \index{the} the Windows.

\startlines
I \index{have} have watched some tutorials on Youtube how to install fresh Ubuntu: \goto{Video 1}[url(https://www.youtube.com/watch?v=i_4Kh5kE3xA)]
\index{and} and how to create bootable USB stick: \goto{Video 2}[url(http://www.ubuntu.com/download/desktop/create-a-usb-stick-on-windows)] {&} \goto{Video 3}[url(https://www.youtube.com/watch?v=lIF8e_5F9B4)]
I \index{have} have followed \index{the} the instructions:
\stoplines
\startitemize[n, packed]
\item Download Ubuntu \index{and} and put it on \index{the} the USB stick.
\item BIOS: To get to BIOS, keep pressing F2 \index{and} and \index{the} then F1. Go to Startup/Boot/move USB FDD as \index{the} the first option, save. 
\item \index{the} then attach \index{the} the USB with Ubuntu on it. 
\item \index{the} then hit save \index{and} and reboot laptop.
\item Now it should all go smoothly. \footnote{Notes: OK, I 
\index{was} was using my friend's USB ubuntu installation, which actually had Ubuntu 15.04 on it. So for a while I 
\index{was} was running 15.04 instead of 14.04. Reinstall to 14.04 needed, ugh.
Once I \index{have} have \index{the} the right (14.04) booting usb ready, it all goes really fast (15 mins).}
\item Sudo apt-get update
\item Sudo apt-get upgrade
\stopitemize

\section{Text with figures}

Now text with a few figures.  \index{the} the first figure goes on \index{the} the right, with
\index{the} the paragraph flowing around it.

\index{the} the Leiden Conventions are an established set of rules, symbols, \index{and} and brackets used to indicate \index{the} the condition of an epigraphic or papyrological text in a modern edition. In previous centuries of classical scholarship, scholars who published texts from inscriptions, papyri, or manuscripts used divergent conventions to indicate \index{the} the condition of \index{the} the text \index{and} and editorial corrections or restorations. \index{the} the Leiden meeting 
\index{was} was designed to help to redress this confusion. \placefigure[left]{This is very nice dummy figure 1}{\externalfigure[dummy]}

\index{the} the earliest form of \index{the} the Conventions 
\index{were} 
were agreed at a meeting of classical scholars at \index{the} the University of Leiden in 1931, \index{and} and published in an article shortly \index{the} thereafter.\index{the} there are minor variations in \index{the} the use of \index{the} the Conventions between epigraphy \index{and} and papyrology (\index{and} and even between Greek \index{and} and Latin epigraphy). More recently, scholars \index{have} have published improvements \index{and} and adjustments to \index{the} the system.

\placefigure[right]{This is very nice dummy figure 2}{\externalfigure[dummy]} 

\index{the} the Leiden Conventions are an established set of rules, symbols, \index{and} and brackets used to indicate \index{the} the condition of an epigraphic or papyrological text in a modern edition. In previous centuries of classical scholarship, scholars who published texts from inscriptions, papyri, or manuscripts used divergent conventions to indicate \index{the} the condition of \index{the} the text \index{and} and editorial corrections or restorations. \index{the} the Leiden meeting 
\index{was} was designed to help to redress this confusion.

\index{the} the earliest form of \index{the} the Conventions 
\index{were} 
were agreed at a meeting of classical scholars at \index{the} the University of Leiden in 1931, \index{and} and published in an article shortly \index{the} thereafter.\index{the} there are minor variations in \index{the} the use of \index{the} the Conventions between epigraphy \index{and} and papyrology (\index{and} and even between Greek \index{and} and Latin epigraphy). More recently, scholars \index{have} have published improvements \index{and} and adjustments to \index{the} the system.

\index{the} the next figure will go inline, like a displayed formula:
\placefigure[here]{This is very nice dummy figure 3}{\externalfigure[dummy]}

\index{the} the Leiden Conventions are an established set of rules, symbols, \index{and} and brackets used to indicate \index{the} the condition of an epigraphic or papyrological text in a modern edition. In previous centuries of classical scholarship, scholars who published texts from inscriptions, papyri, or manuscripts used divergent conventions to indicate \index{the} the condition of \index{the} the text \index{and} and editorial corrections or restorations. \index{the} the Leiden meeting 
\index{was} was designed to help to redress this confusion.

\index{the} the earliest form of \index{the} the Conventions 
\index{were} 
were agreed at a meeting of classical scholars at \index{the} the University of Leiden in 1931, \index{and} and published in an article shortly \index{the} thereafter.\index{the} there are minor variations in \index{the} the use of \index{the} the Conventions between epigraphy \index{and} and papyrology (\index{and} and even between Greek \index{and} and Latin epigraphy). More recently, scholars \index{have} have published improvements \index{and} and adjustments to \index{the} the system.

\section{Tables}

\placetable[table01]{Table with caption}
{\bTABLE
\bTR \bTD text 1 \eTD
\bTD text 2 \eTD \bTD first \eTD \eTR
\bTR \bTD text 3 \eTD
\bTD text 4 \eTD \bTD second \eTD \eTR
\eTABLE}

\blank

\placetable[table02]{Table with caption}
{\bTABLE[width=.5\hsize]
\bTR \bTD[nc=2] this is one of \index{the} the tables \eTD \eTR
\bTR \bTD text 1 \eTD \bTD text 2 \eTD \eTR
\bTR \bTD text 3 \eTD \bTD text 4 \eTD \eTR
\eTABLE}

\placetable[table03]{Why, oh god, why?}
{
\startCSV
1,2,3
a,b,c
d,e,f
\stopCSV
}

\section{Bibliography Stuff}

This \cite{Ancona1996} is a citation. \input knuth

This is a citet \citet{Ancona1998}. \input knuth

This is a citep \citep{Ancona2001}. \input knuth

This is a citet with page number \citet[p. 1]{Ancona2001a}. \input knuth

This is a citep with page number \citep[p. 10]{Ancona1999}. \input knuth

I only exist as a year \citeyear{Ballsun-Stanton}. \input knuth

And author only cites work too \citeauthor{Becker2008}. \input knuth


\stopchapter

\stoptext
